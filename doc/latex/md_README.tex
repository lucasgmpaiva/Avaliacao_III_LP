Avaliação que consiste na criação de um programa que realize uma corrida de sapos. A nota da avaliação irá compor a nota da 3ª unidade da matéria de Linguagem de Programação I (LP I).

\subsection*{Desenvolvedor}

Lucas Gabriel Matias Paiva Aluno do 3º Período do curso de Bacharelado em Tecnologia da Informação (B\+TI) no Instituto Metrópole Digital (I\+MD) da Universidade Federal do Rio Grande do Norte (U\+F\+RN)

\subsection*{Para Compilar O Programa}

\$ make  \$ make move

\subsection*{Para Executar O Programa}

\$ ./bin/\+Corrida

\subsection*{Informações}

\paragraph*{-\/$>$ Passo 1 e 2\+:}

Ao iniciar o programa, é feita uma leitura dos arquivos de sapos e pistas. Os resultados da leitura são armazenados em seus vector\textquotesingle{}s respectivos. E o vector de sapos é atribuido a uma corrida. As funções para a leitura do arquivo de sapos e para a leitura do arquivo de pistas se chamam, respectivamente, \char`\"{}ler\+Arquivo\+Sapos e \char`\"{}ler\+Arquivo\+Pistas\char`\"{}, ambas definidas e implementadas em \char`\"{}\hyperlink{opPista_8hpp}{op\+Pista.\+hpp}\char`\"{} e \char`\"{}\hyperlink{opPista_8cpp}{op\+Pista.\+cpp}". 

\paragraph*{-\/$>$ Passo 3\+:}

a) Para visualizar as estatísticas dos sapos, o usuário deve selecionar a opção 4 no menu principal do programa. Fazendo isso, o programa entrará num laço e exibirá as características de cada sapo, sendo que, para isso, foi feita uma sobrecarga do operador \char`\"{}$<$$<$\char`\"{} na classe \hyperlink{classSapo}{Sapo}. 

b) Para visualizar as estatísticas das pistas, o usuário deve selecionar a opção 5 no menu principal do programa. Fazendo isso, o programa entrará num laço e exibirá as características de cada pista, sendo que, para isso, foi feita uma sobrecarga do operador \char`\"{}$<$$<$\char`\"{} na classe \hyperlink{classPista}{Pista}. 

c) Para iniciar uma corrida, o usuário deverá selecionar a opção 1 no menu principal do programa. Assim, o programa \char`\"{}chamará\char`\"{} a função para selecionar pista, denominada \char`\"{}select\+Pista\char`\"{}, definida em op\+Pista (tanto .hpp, quanto .cpp), que exibirá todas as pistas disponíveis e solicitará que o usuário selecione uma de acordo com seu identificador (ID). Fazendo isso, o programa \char`\"{}chamará\char`\"{} a função \char`\"{}realizar\+Corrida\char`\"{} definida e implementada na classe \hyperlink{classCorrida}{Corrida}. E a corrida será realizada. E logo depois, o arquivo de Sapos será reescrito com as novas informações. 

d) Para criar um sapo, o usuário deve selecionar a opção 2 no menu principal do programa. Ao fazer isso, o programa irá executar a função \char`\"{}adicionar\+Sapo\char`\"{} definida na classe \hyperlink{classCorrida}{Corrida}. Em seguida, serão solicitadas as características do sapo, e então o sapo será adicionado ao vector de Sapos e, por consequência à corrida. Por fim, o programa reescreverá o arquivo de Sapos para registrar o novo sapo adicionado. 

e) Para criar uma pista, o usuário deve selecionar a opção 3 no menu principal do programa. Ao fazer isso, o programa irá executar a função \char`\"{}criar\+Pista\char`\"{} definida em op\+Pista (tanto .hpp, quanto .cpp). Em seguida, serão solicitadas o comprimento da pista, e então a pista será adicionada ao vector de Pistas e, por consequência será disponibilizada para ser usada em uma corrida. Por fim, o programa reescreverá o arquivo de pistas para registrar a nova pista adicionada.

\paragraph*{-\/$>$ Passo 4\+:}

a) Ao selecionar para iniciar uma corrida, o programa irá exibir todas as pistas disponíveis para a realização de uma corrida, com o uso da função \char`\"{}select\+Pista\char`\"{} definida em op\+Pista (tanto .hpp, quanto .cpp). Em seguida, é esperado do usuário, o id da pista que ele deseja. 

b) Após selecionada a pista, serão exibidos todos os sapos participantes da corrida para o usuário\+: seus nomes, seus id\textquotesingle{}s e suas potências de salto/pulo. Isso acontece, pois a função \char`\"{}realizar\+Corrida\char`\"{} irá \char`\"{}chamar\char`\"{} a função \char`\"{}mostrar\+Sapos\char`\"{} definida e implementada na classe \hyperlink{classCorrida}{Corrida}. 

c) Enfim, para iniciar a corrida, o usuário deverá apertar enter, e em seguida a corrida começará.

\paragraph*{-\/$>$ Passo 5\+:}

a) Durante a corrida, quando cada sapo pular, serão exibidas, em tela, as informações\+: id, pulo dado e seu nome (não necessariamente nessa ordem). Tudo isso definidio da função \char`\"{}realizar\+Corrida\char`\"{} da classe \hyperlink{classCorrida}{Corrida}. 

b) Assim que cada sapo chegar ao fim da corrida, será alertado que ele cruzou a linha de chegada, e em seguida apenas os sapos restantes irão exibir mensagens na tela. Tudo isso definidio da função \char`\"{}realizar\+Corrida\char`\"{} da classe \hyperlink{classCorrida}{Corrida}. 

c) Por fim, a corrida só acabará quando o último sapo cruzar a linha de chegada. E assim, será exibido, em tela, o ranking completo com a posição de cada sapo ao final da corrida. Tudo isso definidio da função \char`\"{}realizar\+Corrida\char`\"{} da classe \hyperlink{classCorrida}{Corrida}.

\paragraph*{-\/$>$ Passo 6\+:}

A reescrita de cada arquivo (de Sapos e Pistas) é feita\+: após uma corrida ser realizada, após um novo sapo ser criado, após uma nova pista ser criada e, para garantir, ao final do programa é realizada a reescrita. A escrita dos arquivos foi definida e implementada em \char`\"{}func\+Arquivos\char`\"{} (tanto .hpp, quanto .cpp), como \char`\"{}escrever\+Arquivo\+Sapos\char`\"{} e \char`\"{}escrever\+Arquivo\+Pistas\char`\"{}.

\subsection*{Informações Extras}

A forma com a qual está escrita os arquivos de informações (de sapos e pistas), para cada sapo ou pista é\+:

\paragraph*{Sapos\+:}

1ª Linha\+: Id do \hyperlink{classSapo}{Sapo} 

2ª Linha\+: Nome do \hyperlink{classSapo}{Sapo} 

3ª Linha\+: Número de Corridas Realizadas Pelo \hyperlink{classSapo}{Sapo} 

4ª Linha\+: Total de Pulos Dados Pelo \hyperlink{classSapo}{Sapo} 

5ª Linha\+: Potência de Pulo do \hyperlink{classSapo}{Sapo} 

6ª Linha\+: Número de Vitórias do \hyperlink{classSapo}{Sapo} 

7ª Linha\+: Número de Empates do \hyperlink{classSapo}{Sapo}

\paragraph*{Pistas\+:}

1ª Linha\+: Id da \hyperlink{classPista}{Pista} 

2ª Linha\+: Comprimento da \hyperlink{classPista}{Pista}

\subsubsection*{\href{https://github.com/lucasgmpaiva/Avaliacao_III_LP}{\tt Acesse aqui o projeto no Git\+Hub }}